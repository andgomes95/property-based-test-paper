\section{Background} \label{sec:back}

	In this section we present a brief introduction to the basic concepts of Property Based Test (PBT). We start pointing the definition of properties. After, we explain the definition of PBT, comparing it with unity tests and presenting libraries that use this type of test.
	  
	\subsection{Property}
	
	Properties are general rules that describe the behavior of a algorithm and most be applied for each code snippet input and output~\cite{9781680506211}. The properties are written and operated in a high level, describing generically the result of each operation.
	The work of ~\cite{fink1997property}, defines that the first step to making a PTB is to select a property of a generic set of properties, or choose a program specifics.
	 
	\subsection{Property Based Test}

	PBT has as characteristic to specify many properties that aim user's garantee that a algorithm answers to a given property ~\cite{fink1997property}. Besides that, PBT aims to take the principles and expectations towards the code behavior and use as test, instead of specifics examples. The quality of unity test are determinated by the programmer's hability in thinking coverage cases, which is given by his expertise, attention to details and code's general knowledge. The expectation in PBT is to define a generic property that could cover any entry of the code, and the most uncertain and unknown scenarios could be detected.

	\begin{table}[ht]
		\begin{center}
		    \begin{tabular}{|r|r|r|}
		    \hline
		    \textbf{Library} & \textbf{Language} & \textbf{Reference} \\ \hline
		    QuickCheck & Haskell & ~\cite{quickcheck} \\ \hline
		    PropEr & Erlang & ~\cite{proper} \\ \hline
		    SwiftCheck & Swift & ~\cite{swiftcheck} \\ \hline
		    JSverify & JavaScript & ~\cite{jsverify} \\ \hline
		    FsCheck & .NET & ~\cite{fscheck} \\ \hline
		    JUnit-QuickCheck & Java & ~\cite{junit} \\ \hline
		    Rantly & Ruby & ~\cite{rantly} \\ \hline
		    Stream Data & Elixir & ~\cite{stream} \\ \hline
		    GOlang Property TestER (GopTer) & GO & ~\cite{gopter} \\ \hline
		    Hypothesis & Python/Ruby/Java & ~\cite{hyp} \\ \hline 
	    	\end{tabular}
		\end{center}
		\label{tab:libraries}
		\caption{Avalible PBT libraries}
	\end{table}

	The tools for PBT has as principle the generation of random entries that exercise code different parts. In Table ~\ref{tab:libraries} lists the available libraries for PBT in different programming languages. Between these libraries the most used is QuickCheck for Haskell. QuickCheck works by random testing the program properties ~\cite{quickcheck}. Thus, the programmer provides a program specification in the form properties which function should satisfy, then QuickCheck run a multiple randomly generated test cases. This library served as base for other frameworks and libraries as: ProPer, SwiftCheck, JSVerify, FsCheck, JUnit-QuickCheck e Rantly. QuickCheck works by providing a code specification, in the form of properties which functions should satisfy,  ~\cite{quickcheck}. Other proposed libraries which are not based in QuickCheck is also available as StreamData, GOland Property TestER and Hypothesis. 