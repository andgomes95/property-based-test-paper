\section{Introdução}\label{sec:introducao}
	
	%% falo do aumento do número de acessos a informação e tecnlogia
	A expansão das tecnologias da informação e comunicação (TICs), proporcionou a população em geral um maior contato com dispositivos eletrônicos \footnote{https://www.telesemana.com/futurecom/pt/2014/10/08/mercado-nacional-de-tics-e-perspectivas-de-expansao/} como celulares, computadores e \textit{video games}. Estes dispositivos e outros são controlados pelos chamados \textit{softwares}, os quais proporcionam funcionalidades a estes limitadas de acordo com o \textit{hardware}. Estima-se que o valor despendido com \textit{software} no ano de 2019 será de 201 bilhões de dólares \footnote{https://www.itforum365.com.br/digital/transformacao-digital-esta-por-tras-da-expansao-do-mercado-de-software-veja-8-tendencias/}, justificando assim a utilização e a importância deste ramo.

	%% com expansão da utilização do software a necessidade de se testar softwares para que haja qualidade dos mesmos
	Com o crescimento do indústria do \textit{software} houve a necessidade de garantir a qualidade destes produtos ao usuário final, não apenas em grandes companhias, em programas de larga escala, mas também em \textit{softwares} mais simples. O custo causado por falhas em programas para dispositivos chegou em 1,1 trilhão de dólares em 2016 \footnote{http://www.base2.com.br/2017/03/20/falhas-em-software-provocaram-prejuizos-de-us-1-1-trilhao-em-2016/}, sendo que o montante consumido poderia ter sido evitado com teste de software.

	%% pensando nisso foram feitos diversos tipos de modelos para testes
	Pensando nestes fatos, pesquisadores e empresas desenvolvem e aplicam a cada ano diversos tipos de teste de software. Dentre eles temos, testes baseado em modelos, teste funcional, teste de integração, teste de regressão, entre outros, os quais podem ser aplicados em diferentes etapas do processo de desenvolvimento de software, contudo, possuem o mesmo objetivo final \cite{paiva2016aplicaccao}.

	%% dentre eles temos o property based testing, explico o que é o property based testing
	Uma das abordagens para assegurar a qualidade do \textit{software} presentes na literatura atualmente é o teste baseado em propriedades, o qual aproveita os princípios e expectativas em relação ao comportamento do código e utiliza estes para testar, ao invés de aplicar exemplos específicos \cite{fink1997property}. Em outras palavras, constrói-se os casos de teste de maneira que estes revelam a presença de falhas que não são reveladas pela execução direta do código. 

	%% Neste estudo, visamos dar uma visão geral sobre teste baseado em propriedades...
	Em virtude destes fatos, neste estudo, objetivamos dar uma visão geral sobre o teste baseado em propriedades fazendo uma investigação das pesquisas realizadas no tema, dado que de acordo com nossos conhecimentos existem poucos estudos que realizam tal. Para isso, utilizamos em nossa metodologia a técnica de revisão sistemática. A revisão sistemática é um método que tem como objetivo mensurar a extensão de um determinado tema na literatura por meio de uma busca de trabalhos da área \cite{petersen2008systematic}.

	O restante do artigo está dividido da seguinte forma. Na seção \ref{sec:trabalhos_relacioandos}, é apresentada a fundamentação teórica para o trabalho. Na seção \ref{sec:revisao_sistematica}, é apresentada a revisão sistemática deste trabalho, como protocolos e os passos para realizamos o processo. Em sequência, na seção \ref{sec:resultados} são apresentados os resultados de nossa pesquisa. Por fim, na seção \ref{sec:conclusao} é apresentada a conclusão de nossa trabalho e algumas propostas de trabalhos futuros.