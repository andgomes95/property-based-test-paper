\section{Revisão Sistemática} \label{sec:revisao_sistematica}


Nessa seção, apresentaremos uma revisão sistemática que foi desenvolvida no trabalho. 

\subsection{Protocolo}

Como citado anteriormente, a revisão sistemática tem como objetivo fornecer uma visão de uma certa área utilizando um processo rigoroso de coleta e avaliação \cite{petersen2008systematic}. Espera-se obter como resultado o entendimento das tendências da área sobre o tema pesquisado.  A definição do objetivo de pesquisa e das questões de pesquisa são apresentados a seguir. 

\begin{itemize}
    \item \textbf{Objetivo:} Identificar estudos que falem sobre o “property-based testing” e as aplicações até o momento.
    \item \textbf{Questões de pesquisa:} 
    \begin{enumerate}
        \item Quais são os estudos atuais sobre “property-based testing”?
        \item Quais são as ferramentas que suportam esse tipo de teste?
        \item Qual vertente do "property-based testing" está sendo mais utilizada?
        \item Em qual domínio está sendo mais aplicado?
        \item Quais são as vantagens e desvantagens da utilização do "property-based testing"?
        \item Quais são os maiores desafios encontrados pelos pesquisadores nessa área?
    \end{enumerate}
\end{itemize}

Partindo do objetivo e das questões de pesquisa foram definidas estratégias para realizar a busca e a seleção dos estudos.

\subsection{Busca e seleção de estudos}

A seleção dos estudos e a estratégia de busca foram definidos de acordo com os tópicos abaixo: 

\begin{itemize}
    \item \textbf{Critérios para seleção das fontes:} Os critérios empregados para a seleção das fontes utilizadas no trabalho seguiram as principais conferências, periódicos e simpósios em que são publicados estudos da área de Teste de Software.
    
    
    \item \textbf{Métodos de pesquisa:} Será utilizada uma busca automática na qual será criada uma \textit{string} de busca e a mesma será executada nas fontes selecionadas.
    
    
    \item \textbf{Palavras-chave:} \textit{test, software testing, property-based testing}
    
    
    \item \textbf{String de busca:} ("property-based testing" OR "property based testing" OR "property based test" OR "property-based test") 
    
    
    \item \textbf{Listagem das Fontes Selecionadas:} Springer\footnote{\url{http://www.springerlink.com/}}, IEEE\footnote{\url{http://ieeexplore.ieee.org}}, ACM\footnote{\url{http://portal.acm.org/dl.cfm}} e Elsevier-Science Direct\footnote{\url{http://www.sciencedirect.com/}}
\end{itemize}

\subsection{Critérios para seleção dos estudos}

Para selecionar os estudos, foram definidos alguns critérios de inclusão e exclusão que são apresentados a seguir:

\begin{itemize}
    \item \textbf{Critério 1:} artigos que tratem de teste de software
    
    
    
    \item \textbf{Critério 2:} artigos que tratem de \textit{Property-based}
    
    
    \item \textbf{Critério 3:} artigos completos em inglês ou português
    
    
    \item \textbf{Critério 4:} artigos com texto completo disponível na \textit{Web}
    
    
    \item \textbf{Critério 5:} artigos que possuem resumo
\end{itemize}


\subsection{Processo de seleção e extração dos estudos}

O processo para selecionar os estudos relevantes foi realizado na etapa de seleção.

\begin{itemize}
    \item \textbf{Processo de seleção:} Nesse processo, são identificados os estudos relevantes, de
acordo com a leitura do título, do resumo e das palavras-chave. Vale a pena ressaltar
que nessa etapa também são excluídos os trabalhos duplicados. A relevância dos artigos foi feita com base na
leitura de todo o artigo selecionado para verificar se este atende aos requisitos estabelecidos previamente.
 
\end{itemize}

A Tabela \ref{tab:resultados} apresenta os resultados da revisão sistemática em números de acordo com as bases selecionadas. 

\begin{table}[htbp]
\centering
\caption{Tabela Resultados Revisão Sistemática}
\label{tab:resultados}
\begin{tabular}{|c|c|c|c|}
\hline
\textbf{Fontes} & \textbf{Retornados} & \textbf{Seleção} \\ \hline
IEEE & 33 &  19  \\ \hline
Springer & 130 &  16  \\ \hline
ACM & 159 &  0  \\ \hline
Science-Direct & 26 & 1  \\ \hline
\textbf{TOTAL} & \textbf{348} & \textbf{36} \\ \hline
\end{tabular}
\end{table}
